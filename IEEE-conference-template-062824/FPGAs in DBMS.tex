\documentclass[conference]{IEEEtran}
%\IEEEoverridecommandlockouts
% The preceding line is only needed to identify funding in the first footnote. If that is unneeded, please comment it out.
%Template version as of 6/27/2024

\usepackage{cite}
\usepackage[ngerman]{babel}
\usepackage{amsmath,amssymb,amsfonts}
\usepackage{algorithmic}
\usepackage{graphicx}
\usepackage{textcomp}
\usepackage{xcolor}
%\usepackage{todonotes}
\newcommand{\todo}[1]{\textcolor{red}{TODO: #1}}
\def\BibTeX{{\rm B\kern-.05em{\sc i\kern-.025em b}\kern-.08em
    T\kern-.1667em\lower.7ex\hbox{E}\kern-.125emX}}
\begin{document}

\title{FPGAs in DBMS
}

\author{\IEEEauthorblockN{1\textsuperscript{st} Felix Grenzing}
    \IEEEauthorblockA{\textit{Universität Hamburg} \\
        Hamburg, Deutschland\\
        felix.grenzing@studium.uni-hamburg.de}
}

\maketitle

\begin{abstract}
    Abstract here.
\end{abstract}

\begin{IEEEkeywords}
    FPGAs, DBMS, Hardware Acceleration
\end{IEEEkeywords}

\section{Einführung}

\section{Begriffe}


\subsection{FPGAs}


\subsection{Einsatzmöglichkeiten}

\subsection{Pipelining und Datenparallelität}

\subsection{Partial Reconfiguration}

\subsection{BitWeaving}


\section{Einordnung der Situation}
\subsection{Pessimismus}

\subsection{Optimismus}

\section{Anderes Paper}

\subsection{Architektur}

\subsection{Verschiedene Ansätze}

\subsection{Ergebnisse}

\section{Andere Ansätze}

\subsection{IBEX}

\subsection{REGEXP\_LIKE}

\section{The Road that lies ahead}

\begin{thebibliography}{00}
    \bibitem{b1} G. Eason, B. Noble, and I. N. Sneddon, ``On certain integrals of Lipschitz-Hankel type involving products of Bessel functions,'' Phil. Trans. Roy. Soc. London, vol. A247, pp. 529--551, April 1955.
    \bibitem{b2} J. Clerk Maxwell, A Treatise on Electricity and Magnetism, 3rd ed., vol. 2. Oxford: Clarendon, 1892, pp.68--73.
\end{thebibliography}


\end{document}
